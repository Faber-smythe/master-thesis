\section{Related Work}

\textbf{VR implementations} Virtual Reality (VR) has become a commercially available solution for some time. 

Google Street View launched in 2007 as a free service using panoramic imagery, and set the standard for immersive experiences \cite{}. First using Adobe Flash, it now relies on a private javascript engine.

Today, VR solutions are implemented in most of the 3D industry : softwares like Unity \cite{}, Unreal Engine \cite{}, Blender \cite{}, and platforms like steam all added their own VR support feature in the last 5 years. 

\textbf{Online Immersive Environments} Several online solutions offer to create virtual tours for their customers. 

My360tours is a software provider owned by Plush Global Media. It offers to manage virtual tours and various features like floor plans, web embedding, pop ups, and contact forms for a monthly subscription. \cite{}

Concept3D is a company specialized in 360 tours and 3D maps of large facilitie, providing tailor-made solutions. \cite{}

Real Tour Vision (Inc.) provides services specialized in real estate, including virtual home staging, a virtual tour management software, and photography services \cite{}. It is a Google Trusted Agency, and their 3D visits can be available on Google Street View. 

Kuula is a platform allowing users to create their virtual tour by uploading their 360° images. They have a free of charge version, but it doesn't include any edition feature. \cite{}

\textbf{Opensource projects} Free options also allow the development of VR applications.

GuriVR is a VR editor made as part of the Knight-Mozilla Fellowships programm \cite{}. This editor is able to convert a text description of a 3D scene into  VR-Ready scene. It is using PreactJS for the front-end, Node for the back-end, MongoDB as database, and Google Street View API for the 3D rendering. \cite{}

Playcanvas is a web-oriented game engine with support for AR and VR. It can be used as free library, but it also has an online editor with various subscription offers. \cite{} It relies on the WebGL API for rendering, AmmoJS for physics and Web Audio API for sound.

A-Frame is a WebGL framework made for VR and MR (Mixed Reality) experiences. It provides a component-based architecture built on WebGL, WebVR and on the ThreeJS library. \cite{}

BabylonJS is a web rendering engine developed by Microsoft and available as a javascript library. It supports WebGL2, WebGPU, and relies on WebXR for VR and AR applications. It also offers numerous tools for collaboration, shaders, GUI and debugging. \cite{}

\textbf{Metaverse} Many huge companies have started investing in a metaverse solution.

Meta (formerly Facebook) has officially published Horizon World in December 2021. It is a multiplayer sandbox platform allowing users to create worlds, avatars, environments, games, and sharing content. Unlike other metaverse solutions, Horizon Worlds doesn't rely on blockchain or NFT at the time of writing.\cite{}

MetaHero is a project relying on 3D scanning technology to bring real worlds objects into the virtual environment. It is developed by Wolf Digital World and uses a cryptocurrency called \$HERO for virtual real estate, scanning, NFT and licensing. \cite{}

Star Atlas is a science-fiction game-oriented metaverse, developed by Sperasoft, incorporating cryptocurrencies and NFTs. It is using Unreal Engine 5 as rendering engine. \cite{}

Fortnite, a 350 million players battle royale game, is growing into a metaverse, but Epic Games also entered a long-term partnership with Lego around a metaverse aimed at children and families. \cite{}










